\section{Discussion}

In our exploration of timing side-channels, a repeated observation was that vulnerabilities were introduced from attempts at optimization, whether by the programmer or the compiler. Vulnerabilities could be removed only by ensuring that the program executes in the same amount of time regardless of the value of any secret data. 
This phenomenon builds a kind of philosophical construct similar to the Heisenberg
uncertainty principle wherein the limitations of our physical reality are stopping
us from being both optimized (take less time in most cases) and, for example, 
completely computationally private (as observed in \cite{oberg2014leveraging}). The
case in which the least side-channel activity is present is the case in which the 
lengths of all the possible routes of execution of the main-channel are of similar
size, and thus observation becomes noisy in the presence of other physical factors.
This, unsurprisingly, is the case in which there is no optimization made and all the
lengths are similar to the longest one.


%to be cleaned up

%mainly discuss optimization versus safety 
%optimization on many levels, programmer versus compiler